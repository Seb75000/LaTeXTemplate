\chapter*{Annexe 1}

\begin{lstlisting}[language=python, basicstyle=\ttfamily\small, frame=single]

# Importation des modules necessaires
import random
import time

# Définition d'une fonction d'exemple
def lorem_ipsum_function(param1, param2):
    """
    Cette fonction est un exemple de code lorem ipsum.
    Elle ne fait rien de particulier mais illustre une structure de code.
    """
    result = param1 + param2
    for i in range(10):
        result += random.randint(1, 10)
        time.sleep(0.1)  # Simule un temps de traitement

    if result % 2 == 0:
        print("Le résultat est pair.")
    else:
        print("Le résultat est impair.")

    return result

# Définition d'une classe d'exemple
class LoremIpsumClass:
    def __init__(self, value):
        self.value = value

    def process_value(self):
        self.value *= 2
        print(f"La valeur traitée est : {self.value}")

    def display_value(self):
        print(f"La valeur actuelle est : {self.value}")

# Utilisation de la fonction et de la classe
if __name__ == "__main__":
    # Appel de la fonction
    result = lorem_ipsum_function(5, 10)
    print(f"Résultat de la fonction : {result}")

    # Création d'une instance de la classe
    lorem_instance = LoremIpsumClass(20)
    lorem_instance.display_value()
    lorem_instance.process_value()
    lorem_instance.display_value()

    # Boucle d'exemple
    for i in range(5):
        print(f"Itération {i} : {random.choice(['foo', 'bar', 'baz'])}")

    # Condition d'exemple
    if result > 50:
        print("Le résultat est supérieur à 50.")
    else:
        print("Le résultat est inférieur ou égal à 50.")

\end{lstlisting}